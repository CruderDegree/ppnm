\documentclass[twocolumn, a4paper]{article}
\usepackage{graphicx}
\usepackage{amssymb}
\usepackage{amsmath}

\title{The natural logarithm function}
\author{Tobias Rasmussen}

\begin{document}

\maketitle

\section{Introduction}
The natural logarithm of a number is its logarithm to the base of Euler's number $e \approx 2.718281828459$. The natural logarithm is usually referred to as either $\ln(x)$ or $\log_e(x)$.

The natural logarithm maps multiplication into addition as all other logarithms do:
\begin{equation}
	\ln xy = \ln x + \ln y
\end{equation}

\section{Definition}
The natural logarithm function can be defined for some $x\in \mathbb{R}_{>0}$ by the integral
\begin{equation}
	\ln(x) = \int_1^x \frac{1}{t} dt
\end{equation}

If $x<1$, this integral is negative.

\begin{figure}
	\input{plot.tex}
	\caption{Graph of the natural logarithm function along with some tabulated values.}
	\label{fig:plot}
\end{figure}

\section{Properties}
The natural logarithm satisfies the following properties, among others:
\begin{align}
	&\ln 1 = 0 \\
	&\ln e = 1 \\
	&\ln(xy) = \ln x + \ln y, \quad x,y > 0 \\
	&\lim_{x\rightarrow 0} \frac{\ln(1+x}{x} = 1
\end{align}

\end{document}
